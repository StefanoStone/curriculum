\documentclass[10pt, letterpaper]{article}

% Packages:
\usepackage[
    ignoreheadfoot, % set margins without considering header and footer
    top=2 cm, % seperation between body and page edge from the top
    bottom=2 cm, % seperation between body and page edge from the bottom
    left=2 cm, % seperation between body and page edge from the left
    right=2 cm, % seperation between body and page edge from the right
    footskip=1.0 cm, % seperation between body and footer
    % showframe % for debugging 
]{geometry} % for adjusting page geometry
\usepackage{titlesec} % for customizing section titles
\usepackage{tabularx} % for making tables with fixed width columns
\usepackage{array} % tabularx requires this
\usepackage[dvipsnames]{xcolor} % for coloring text
\definecolor{primaryColor}{RGB}{0, 79, 144} % define primary color
\usepackage{enumitem} % for customizing lists
\usepackage{fontawesome5} % for using icons
\usepackage{amsmath} % for math
\usepackage[
    pdftitle={Stefano Campanella's CV},
    pdfauthor={Stefano Campanella},
    pdfcreator={LaTeX},
    colorlinks=true,
    urlcolor=primaryColor
]{hyperref} % for links, metadata and bookmarks
\usepackage[pscoord]{eso-pic} % for floating text on the page
\usepackage{calc} % for calculating lengths
\usepackage{bookmark} % for bookmarks
\usepackage{lastpage} % for getting the total number of pages
\usepackage{changepage} % for one column entries (adjustwidth environment)
\usepackage{paracol} % for two and three column entries
\usepackage{ifthen} % for conditional statements
\usepackage{needspace} % for avoiding page brake right after the section title
\usepackage{iftex} % check if engine is pdflatex, xetex or luatex
\usepackage{etoolbox} % For \patchcmd
\usepackage{lastpage} % For LastPage reference

% Ensure that generate pdf is machine readable/ATS parsable:
\ifPDFTeX
    \input{glyphtounicode}
    \pdfgentounicode=1
    % \usepackage[T1]{fontenc} % this breaks sb2nov
    \usepackage[utf8]{inputenc}
    \usepackage{lmodern}
\fi



% Some settings:
\AtBeginEnvironment{adjustwidth}{\partopsep0pt} % remove space before adjustwidth environment
\pagestyle{empty} % no header or footer
\setcounter{secnumdepth}{0} % no section numbering
\setlength{\parindent}{0pt} % no indentation
\setlength{\topskip}{0pt} % no top skip
\setlength{\columnsep}{0cm} % set column seperation
\makeatletter
\let\ps@customFooterStyle\ps@plain % Copy the plain style to customFooterStyle
\patchcmd{\ps@customFooterStyle}{\thepage}{
    \color{gray}\textit{\small Stefano Campanella - Pagina \thepage{} di \pageref*{LastPage}}
}{}{} % replace number by desired string
\makeatother
\pagestyle{customFooterStyle}

\titleformat{\section}{\needspace{4\baselineskip}\bfseries\large}{}{0pt}{}[\vspace{1pt}\titlerule]

\titlespacing{\section}{
    % left space:
    -1pt
}{
    % top space:
    0.3 cm
}{
    % bottom space:
    0.1 cm
} % section title spacing

\renewcommand\labelitemi{$\circ$} % custom bullet points
\newenvironment{highlights}{
    \begin{itemize}[
        topsep=0.10 cm,
        parsep=0.10 cm,
        partopsep=0pt,
        itemsep=0pt,
        leftmargin=0.4 cm + 10pt
    ]
}{
    \end{itemize}
} % new environment for highlights

\newenvironment{highlightsforbulletentries}{
    \begin{itemize}[
        topsep=0.10 cm,
        parsep=0.10 cm,
        partopsep=0pt,
        itemsep=0pt,
        leftmargin=10pt
    ]
}{
    \end{itemize}
} % new environment for highlights for bullet entries


\newenvironment{onecolentry}{
    \begin{adjustwidth}{
        0.2 cm + 0.00001 cm
    }{
        0.2 cm + 0.00001 cm
    }
}{
    \end{adjustwidth}
} % new environment for one-column entries

\newenvironment{twocolentry}[2][]{
    \onecolentry
    \def\secondColumn{#2}
    \setcolumnwidth{\fill, 4.5 cm}
    \begin{paracol}{2}
}{
    \switchcolumn \raggedleft \secondColumn
    \end{paracol}
    \endonecolentry
} % new environment for two-column entries

\newenvironment{header}{
    \setlength{\topsep}{0pt}\par\kern\topsep\centering\linespread{1.5}
}{
    \par\kern\topsep
} % new environment for the header

\newcommand{\placelastupdatedtext}{% \placetextbox{<horizontal pos>}{<vertical pos>}{<stuff>}
  \AddToShipoutPictureFG*{% Add <stuff> to current page foreground
    \put(
        \LenToUnit{\paperwidth-2 cm-0.2 cm+0.05cm},
        \LenToUnit{\paperheight-1.0 cm}
    ){\vtop{{\null}\makebox[0pt][c]{
        \small\color{gray}\textit{Ultimo aggiornamento Gennaio 2026}\hspace{\widthof{Ultimo aggiornamento Gennaio 2026}}
    }}}%
  }%
}%

% save the original href command in a new command:
\let\hrefWithoutArrow\href

% new command for external links:
\renewcommand{\href}[2]{\hrefWithoutArrow{#1}{\ifthenelse{\equal{#2}{}}{ }{#2 }\raisebox{.15ex}{\footnotesize \faExternalLink*}}}


\begin{document}
    \newcommand{\AND}{\unskip
        \cleaders\copy\ANDbox\hskip\wd\ANDbox
        \ignorespaces
    }
    \newsavebox\ANDbox
    \sbox\ANDbox{}

    \placelastupdatedtext
    \begin{header}
        \textbf{\fontsize{24 pt}{24 pt}\selectfont Stefano Campanella}

        \vspace{0.3 cm}

        \normalsize
        \mbox{\hrefWithoutArrow{mailto:stefanocamp187@gmail.com}{\color{black}{\footnotesize\faEnvelope[regular]}\hspace*{0.13cm}stefanocamp187@gmail.com}}%
        \kern 0.25 cm%
        \AND%
        \kern 0.25 cm%
        \mbox{\hrefWithoutArrow{https:/stefanostone.github.io//}{\color{black}{\footnotesize\faLink}\hspace*{0.13cm}stefanostone.github.io}}%
        \kern 0.25 cm%
        % \AND%
        % \kern 0.25 cm%
        % \mbox{\hrefWithoutArrow{tel:+39-345-45-78-280‬}{\color{black}{\footnotesize\faPhone*}\hspace*{0.13cm}+39 345 45 78 280}}%
        % \kern 0.25 cm%
        \AND%
        \kern 0.25 cm%
        \mbox{\hrefWithoutArrow{https://linkedin.com/in/stefano-campanella-b26998268}{\color{black}{\footnotesize\faLinkedinIn}\hspace*{0.13cm}Stefano Campanella}}%
        \kern 0.25 cm%
        \AND%
        \kern 0.25 cm%
        \mbox{{\color{black}\footnotesize\faMapMarker*}\hspace*{0.13cm}Lugano, Switzerland}%
        \kern 0.25 cm%
        % \AND%
        % \kern 0.25 cm%
        % \mbox{\hrefWithoutArrow{https://github.com/StefanoStone}{\color{black}{\footnotesize\faGithub}\hspace*{0.13cm}StefanoStone}}%
    \end{header}

    \vspace{0.3 cm - 0.3 cm}
    
    \section{Profilo}

        \begin{onecolentry}
             Ho una solida formazione accademica in Ingegneria del software e dei dati, nonché in Sicurezza dei sistemi software, che mi ha permesso di affrontare le moderne sfide di sviluppo con competenza tecnica e pensiero strategico. La mia esperienza spazia dalla ricerca e dal tutoraggio in ambito accademico alla progettazione e alla realizzazione di soluzioni software efficaci in ambito industriale. Combino solide capacità di sviluppo con una chiara comprensione delle esigenze degli utenti, traducendo costantemente requisiti complessi in sistemi affidabili e di alta qualità. Spinto da una passione per il design e i videogiochi, pongo una forte enfasi sull'usabilità dei sistemi, cercando modi intuitivi e coinvolgenti per modellare l'esperienza dell'utente nei sistemi che sviluppo.
        \end{onecolentry}

    
    \section{Competenze}

        \begin{onecolentry}
            Angular - Bash - Continuous Integration/Delivery - Data Mining - Docker - FastAPI - Generative AI/LLM - Git - GraphQL - HTML/CSS - Ionic - Java - JavaScript/TypeScript - Linux - Machine Learning - Mobile Development - NLP - Pharo Smalltalk - Python - Scikit-learn - Weka - XCode - Teamwork - Leadership - Problem-Solving - Public Speaking
        \end{onecolentry}

     
    \section{Esperienza}
        \begin{twocolentry}{
        Lugano, CH
        
        Ott 2025 – Current}
            \textbf{Full-Stack Software Engineer} \textit{Lifeware SA}

            {\footnotesize Tech Stack: Pharo Smalltalk | Extreme Programming}
        \end{twocolentry}

        \vspace{0.10 cm}
                \begin{onecolentry}
            \begin{highlights}
                \item Smalltalk Object-Oriented Full-Stack engineering SaaS
                \item Life Insurance
                \item Extreme Programming
            \end{highlights}
        \end{onecolentry}
        \vspace{0.2 cm}

         \begin{twocolentry}{
        Lugano, CH
        
        Ott 2023 – Sett 2025}
            \textbf{Software Engineer - Ricercatore} \textit{Università della Svizzera italiana}

            {\footnotesize Tech Stack: Python | Pharo Smalltalk | Docker | LaTeX | Git}
        \end{twocolentry}

        \vspace{0.10 cm}
                \begin{onecolentry}
            \begin{highlights}
                \item Ricercatore nel gruppo ``Reverse Engineering, Visualization, Evolution Analysis Lab'' (\href{https://reveal.si.usi.ch/}{REVEAL})
                \item Ricerca nel contesto della caratterizzazione del comportamento degli sviluppatori in progetti Open-Source
                \item Compiti di assistente alla didattica per ``Engineering of Domain Specific Langauges'' e ``Software Atelier 4''
            \end{highlights}
        \end{onecolentry}
        \vspace{0.2 cm}

        \begin{twocolentry}{
        Molise, IT
            
        Mar 2022 - Gen 2023}
            \textbf{Front-End Software Engineer} \textit{Datasound SRL}

            {\footnotesize Tech Stack: Angular | Docker | Java Android | Swift | Git}
        \end{twocolentry}

        \vspace{0.10 cm}
        \begin{onecolentry}
            \begin{highlights}
                \item Manutenzione di un portale web Angular per un software gestionale
                \item Implementazione di un applicativo mobile Ionic per il software di gestione
                \item Integrazione tra il portale web e l'applicativo mobile
            \end{highlights}
        \end{onecolentry}


        \vspace{0.2 cm}
        
        \begin{twocolentry}{
        Molise, IT  
            
        Mar 2022 – Ago 2022}
            \textbf{Software Engineer - Borsa di ricerca} \textit{University of Molise}

            {\footnotesize Tech Stack: Angular | Java SpringBoot | Docker | Git}
        \end{twocolentry}

        \vspace{0.10 cm}
        \begin{onecolentry}
            \begin{highlights}
            \item Implementazione di uno strumento su misura finalizzato all'implementazione di obiettivi e misure di conservazione nei siti ``Natura 2000'' inclusi nelle Riserve e in altre aree governative gestite dall'Arma dei Carabinieri.
            \item ricoperto il ruolo di sviluppatore, designer di UI/UX e product owner
            \end{highlights}
        \end{onecolentry}


        \vspace{0.2 cm}

        \begin{twocolentry}{
        Molise, IT
            
        Ott 2021 – Dic 2021}
            \textbf{Front-End Team Leader / Software Engineer} \textit{Associazione Perfetta Letizia}

            {\footnotesize Tech Stack: Angular | Git }
        \end{twocolentry}

        \vspace{0.10 cm}
        \begin{onecolentry}
            \begin{highlights}
                \item Design di UI/UX per un applicativo mobile
                \item Gestione del team di Front-End e delle interazioni con il commissionante
                \item Implementazione di un'applicazione mobile ibrida in Ionic, per il supporto agli anziani della comunità locale
            \end{highlights}
        \end{onecolentry}



       
    \section{Educazione}   
        \begin{twocolentry}{
                 
        Sett 2022 – Luglio 2023}
            \textbf{Università della Svizzera italiana} \textit{Magistrale in Software and Data Engineering}
        \end{twocolentry}

        \vspace{0.10 cm}
        \begin{onecolentry}
            \begin{highlights}
                \item \textbf{In materia di:} Software engineering, High-performance computing, Artificial Intelligence, Software analytics, Software design and modeling (\href{https://www.usi.ch/en/education/master/software-and-data-engineering}{MSDE})
                %\item GPA: 9.1/10.0
            \end{highlights}
        \end{onecolentry}
        \vspace{0.2 cm}

        \begin{twocolentry}{
                 
        Sett 2021 – Luglio 2023}
            \textbf{Università degli studi del Molise} \textit{Magistrale in Sicurezza dei Sistemi Software}
        \end{twocolentry}

        \vspace{0.10 cm}
        \begin{onecolentry}
            \begin{highlights}
                \item \textbf{In materia di:} Network security, Software security, Computer forensics, Software analytics (\href{https://www2.dipbioter.unimol.it/sicurezza-dei-sistemi-software/}{SSS})
                %\item Grade: 110L/110L 
            \end{highlights}
        \end{onecolentry}
        \vspace{0.2 cm}

        \begin{twocolentry}{
                 
        Sett 2018 – Ott 2021}
            \textbf{Università degli studi del Molise} \textit{Triennale in Informatica}
        \end{twocolentry}

        % \vspace{0.10 cm}
        % \begin{onecolentry}
        %     \begin{highlights}
        %         \item Grade: 108/110L 
        %         \item \textbf{Coursework:} Math, Algorithm, Procedural and object-oriented programming language (C, Java), Logic, Software engineering, Networking, Artificial intelligence 
        %     \end{highlights}
        % \end{onecolentry}
        

    
  
    \section{Progetti}

        \begin{twocolentry}{

        2025}
            \textbf{Interfaccia Python per CLOC (\href{https://github.com/USIREVEAL/pycloc}{pycloc})}
        \end{twocolentry}

        \vspace{0.10 cm}
        \begin{onecolentry}
            \begin{highlights}
                \item Sviluppo Open-Source di un interfaccia per eseguire comandi CLOC tramite codice Python
            \end{highlights}
        \end{onecolentry}


        \vspace{0.2 cm}

        \begin{twocolentry}{
            
            
        2024-2025}
            \textbf{Applicativo per la collezione di dati in larga scala per progetti Git e interazioni GitHub (\href{https://github.com/USIREVEAL/gitminer}{gitminer})}
        \end{twocolentry}

        \vspace{0.10 cm}
        \begin{onecolentry}
            \begin{highlights}

                \item Sviluppo di un applicativo per la collezione di dati large-scale da progetti disponibili su GitHub
                \item L'applicativo gestisce attività in Git (es. Commits) e GitHub (es. Issues)
                \item Implementato in Pharo Smalltalk e reso disponibile con Docker
            \end{highlights}
        \end{onecolentry}


        \vspace{0.2 cm}
        
        
        \begin{twocolentry}{
            
            
        2023}
            \textbf{Prototipo di modello per l'automazione di Testing su Videogiochi}
        \end{twocolentry}

        \vspace{0.10 cm}
        \begin{onecolentry}
            \begin{highlights}
                \item Sviluppo di un prototipo per automatizzare la replicazione di gameplay tramite video raccolti da Twitch.tv utilizzando algoritmi di Machine Learning
                \item Pubblicazione del lavoro nel Workshop ``fase4games2024'' (colocato con FSE24)
            \end{highlights}
        \end{onecolentry}


        \vspace{0.2 cm}

        \begin{twocolentry}{
            
            
        2021}
            \textbf{Applicativo Mobile per la Misurazione della Pressione Sanguinia}
        \end{twocolentry}

        \vspace{0.10 cm}
        \begin{onecolentry}
            \begin{highlights}
                \item Sviluppo di un modello di Machine Learning per la misurazione della pressione sanguignia tramite dati ECG raccolti con Smartwatch
                \item Design di un prototipo di applicativo Android per la misurazione della pressione sanguignia
            \end{highlights}
        \end{onecolentry}

    
    
      \section{Pubblicazioni}

      
        \begin{samepage}
            \begin{twocolentry}{
                July 2025
            }
                \textbf{Fine-Grained Developer Reification}

                \vspace{0.10 cm}

                \mbox{\textbf{Stefano Campanella}}
            \end{twocolentry}


            \vspace{0.10 cm}

            \begin{onecolentry}
        \href{https://doi.org/10.1145/3696630.3731463}{10.1145/3696630.3731463}
            \end{onecolentry}
        \end{samepage}

        \vspace{0.2 cm}
        
        \begin{samepage}
            \begin{twocolentry}{
                Ottobre 2024
            }
                \textbf{Hidden in the Code: Visualizing True Developer Identities}

                \vspace{0.10 cm}

                \mbox{\textbf{Stefano Campanella}}, \mbox{Michele Lanza}
            \end{twocolentry}


            \vspace{0.10 cm}

            \begin{onecolentry}
        \href{https://doi.org/10.1109/VISSOFT64034.2024.00013}{10.1109/VISSOFT64034.2024.00013}
            \end{onecolentry}
        \end{samepage}

        \vspace{0.2 cm}

        \begin{samepage}
        \begin{twocolentry}{
            Luglio 2024
        }
            \textbf{Towards the Automatic Replication of Gameplays \\ to Support Game Debugging}

            \vspace{0.10 cm}

             \mbox{\textbf{Stefano Campanella}}, \mbox{Emanuela Guglielmi}, \mbox{Rocco Oliveto}, \\ \mbox{Gabriele Bavota}, \mbox{Simone Scalabrino}
        \end{twocolentry}


        \vspace{0.10 cm}

        \begin{onecolentry}
    \href{https://doi.org/10.1145/3663532.3664465}{10.1109/3663532.3664465}
        \end{onecolentry}
    \end{samepage}


    
    \section{Lingue}
        
        \begin{onecolentry}
            \textbf{Lingua Madre:} Italiano
        \end{onecolentry}

        \vspace{0.2 cm}

        \begin{onecolentry}
            \textbf{Altre languages:} Inglese (C1)
        \end{onecolentry}

    
    % \section{Soft Skills}
        
    %     \begin{onecolentry}
    %         Teamwork, Leadership, Problem-Solving, Public Speaking
    %     \end{onecolentry}



\end{document}